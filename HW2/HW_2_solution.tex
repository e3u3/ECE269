%%%%%%%%%%%%%%%%% DO NOT CHANGE HERE %%%%%%%%%%%%%%%%%%%% {
\documentclass[12pt,letterpaper]{article}
\usepackage{fullpage}
\usepackage[top=2cm, bottom=4.5cm, left=2.5cm, right=2.5cm]{geometry}
\usepackage{amsmath,amsthm,amsfonts,amssymb,amscd}
\usepackage{lastpage}
\usepackage{enumerate}
\usepackage{fancyhdr}
\usepackage{mathrsfs}
\usepackage{xcolor}
\usepackage{graphicx}
\usepackage{listings}
\usepackage{hyperref}
\usepackage{cases}


\hypersetup{%
colorlinks=true,
linkcolor=blue,
linkbordercolor={0 0 1}
}

\setlength{\parindent}{0.0in}
\setlength{\parskip}{0.05in}
%%%%%%%%%%%%%%%%%%%%%%%%%%%%%%%%%%%%%%%%%%%%%%%%%%%%%%%%%% }

%%%%%%%%%%%%%%%%%%%%%%%% CHANGE HERE %%%%%%%%%%%%%%%%%%%% {
\newcommand\course{ECE 269}
\newcommand\semester{Fall 2019}
\newcommand\hwnumber{\#2}                 % <-- ASSIGNMENT #
\newcommand\NetIDa{Jiaming Lai}           % <-- YOUR NAME
\newcommand\NetIDb{A53314574}           % <-- STUDENT ID #
%%%%%%%%%%%%%%%%%%%%%%%%%%%%%%%%%%%%%%%%%%%%%%%%%%%%%%%%%% }

%%%%%%%%%%%%%%%%% DO NOT CHANGE HERE %%%%%%%%%%%%%%%%%%%% {
\pagestyle{fancyplain}
\headheight 35pt
\lhead{\NetIDa}
\lhead{\NetIDa\\\NetIDb}                 
\chead{\textbf{\Large Assignment \hwnumber}}
\rhead{\course \\ \semester}
\lfoot{}
\cfoot{}
\rfoot{\small\thepage}
\headsep 1.5em
%%%%%%%%%%%%%%%%%%%%%%%%%%%%%%%%%%%%%%%%%%%%%%%%%%%%%%%%%% }

\begin{document}

%%%%%%%%%%%%%%%%%%%%%%%%%%%%%%%%%%%%%%%%%%%%%%%%%%%%%%%%%%
% Problem 1
%%%%%%%%%%%%%%%%%%%%%%%%%%%%%%%%%%%%%%%%%%%%%%%%%%%%%%%%%%
\section*{Problem 1: Convolution as Linea Map}

\textbf{Solution}

\begin{enumerate}[(a)]
    \item 
    Obviously the size of matrice $T$ must be $(N+1)\times(N+1)$. Let $T_{i1},T_{i2}, \ldots ,T_{i(N+1)}$ be the i-th row of matrice $T$.
    Since $y=Tx$, we can get
    \begin{equation}
        v(i-1)=\sum_{j=1}^{N+1} T_{ij}u(j-1)=T_{i1}u(0)+T_{i2}u(1)+ \ldots +T_{i(N+1)}u(N) \label{ProblemF(1)}
    \end{equation}
    Accoring to the convolution function,
    \begin{equation}
        v(i-1)=\sum_{k=-\infty}^{+\infty}h(k)u(i-1-k)=\sum_{k=(i-1-N)}^{i-1}h(k)u(i-1-k) \label{ProblemF(2)} 
    \end{equation}
    Comparing function (1) and function (2), it is easy to conclude that
    \begin{equation}
        T_{ij} = h(i-j), \ where\ 1\leq i \leq (N+1),\ 1\leq j \leq (N+1) \nonumber
    \end{equation}
    So the matrice is a matrice where each element could be represented as $T_{ij} = h(i-j), \ where\ 1\leq i \leq (N+1),\ 1\leq j \leq (N+1)$.
    \item 
    The structure of matrice $T$ could be described as follow:
    \begin{equation}
        T_{i,j} = T_{i+1,j+1} \nonumber
    \end{equation}
\end{enumerate}

%%%%%%%%%%%%%%%%%%%%%%%%%%%%%%%%%%%%%%%%%%%%%%%%%%%%%%%%%%
% Problem 2
%%%%%%%%%%%%%%%%%%%%%%%%%%%%%%%%%%%%%%%%%%%%%%%%%%%%%%%%%%
\section*{Problem 2: Affine Funciton}

\textbf{Solution}

\begin{enumerate}[(a)]
    \item 
    \begin{proof}
        For any $\alpha, \beta \in \mathbb{R}$ and any $x, y \in \mathbb{R}^n$
        \begin{equation}
            \alpha f(x)+\beta f(y) = \alpha (Ax+b) +\beta (Ay+b) = \alpha Ax+ \beta Ay + (\alpha + \beta)b = \alpha Ax+ \beta Ay + b \nonumber
        \end{equation}
        \begin{equation}
            f(\alpha x+\beta y) = A(\alpha x+\beta y)+b =\alpha Ax+ \beta Ay +(\alpha + \beta)b =\alpha Ax+ \beta Ay + b \nonumber
        \end{equation}
        Hence, $\alpha f(x)+\beta f(y)=f(\alpha x+\beta y)$. So function $f(x)= Ax+b$ is affine.
    \end{proof}
    \item 
    \begin{proof}
        First we can show that $b$ is unique. Because $f(0)=b$, so $b$ must be unique, otherwise $f(0)$ will be mapped as different values in
        $\mathbb{R}^m$, which conflicts with the function definition. Then we can show that $A$ is unique. Let function
        \begin{equation}
            g(x) = f(x)-f(0)=Ax+b-b=Ax \nonumber
        \end{equation}
        Suppose $b_1, b_2,\ldots,b_n$ are the basis of
        $\mathbb{R}^n$ and $B=\sum_{i=1}^{n}\alpha_i b_i,\ \alpha_i \in \mathbb{R}$.
        \begin{equation}
            g(B) = g(\sum_{i=1}^{n}\alpha_i b_i) = A(\sum_{i=1}^{n}\alpha_i b_i) = \sum_{i=1}^{n}\alpha_i Ab_i
            =\sum_{i=1}^{n}\alpha_i g(b_i) \nonumber
        \end{equation}
        Suppose $b_i = e_i,\ i=1,2,\ldots,n$, then $g(B)$ above could be represented as
        \begin{equation}
            \begin{split}
                g(B)&= \begin{bmatrix}g(b_1) & g(b_2)&\ldots & g(b_n)\end{bmatrix}\begin{bmatrix}\alpha_1 \\ \alpha_2 \\ \ldots \\\alpha_n\end{bmatrix} \\
                &= \begin{bmatrix}g(e_1) & g(e_2)&\ldots & g(e_n)\end{bmatrix}\begin{bmatrix}x_1 \\ x_2 \\ \ldots \\x_n\end{bmatrix}\\
                &= Ax\\
                \nonumber
            \end{split}
        \end{equation}
        where 
        \begin{itemize}
            \item[] $x\in \mathbb{R}^n$.
        \end{itemize}
        Hence $A$ is unique. So any affine funciton $f$ could be represented uniquely as $f(x)=Ax+b$ for some
        $A\in \mathbb{R}^{m\times n}$ and $b\in \mathbb{R}^{m\times n}$
    \end{proof}
\end{enumerate}

%%%%%%%%%%%%%%%%%%%%%%%%%%%%%%%%%%%%%%%%%%%%%%%%%%%%%%%%%%
% Problem 3
%%%%%%%%%%%%%%%%%%%%%%%%%%%%%%%%%%%%%%%%%%%%%%%%%%%%%%%%%%
\section*{Problem 3: Matrix Multification}

\textbf{Solution}

\begin{enumerate}[(a)]
    \item 
    Suppose
    \begin{equation}
        A = \begin{bmatrix}
            1 & -1 \\
            -1 & 1
        \end{bmatrix}\neq 0 \nonumber
    \end{equation}
    \begin{equation}
        B = \begin{bmatrix}
            1 & 1 \\
            1 & 1
        \end{bmatrix}\neq 0 \nonumber
    \end{equation}
    Then
    \begin{equation}
        AB=0 \nonumber
    \end{equation}
    Hence the statement is incorrect.
    \item 
    Suppose
    \begin{equation}
        A = \begin{bmatrix}
            1 & -1 \\
            1 & -1
        \end{bmatrix}\neq 0 \nonumber
    \end{equation}
    Then
    \begin{equation}
        A^2=0 \nonumber
    \end{equation}
    Hence the statement is incorrect.
    \item 
    Suppose
    \begin{equation}
        A = \begin{bmatrix}
            a_{11} & a_{12} & \ldots & a_{1n} \\
            a_{21} & a_{22} & \ldots & a_{2n} \\
            \vdots & \vdots &  & \vdots \\
            a_{n1} & a_{n2} & \ldots & a_{nn} \\
        \end{bmatrix} \nonumber
    \end{equation}
    Then
    \begin{equation}
        A^TA=
        \begin{bmatrix}
            a_{11}^2 & a_{12}^2 & \ldots & a_{1n}^2 \\
            a_{21}^2 & a_{22}^2 & \ldots & a_{2n}^2 \\
            \vdots & \vdots &  & \vdots \\
            a_{n1}^2 & a_{n2}^2 & \ldots & a_{nn}^2 \\
        \end{bmatrix} \nonumber
    \end{equation}
    So if $A^TA=0$, then $a_{ij}=0,\ 1\leq i \leq n,\ 1\leq j \leq n$. $A=0$. So the statement is correct.
\end{enumerate}

%%%%%%%%%%%%%%%%%%%%%%%%%%%%%%%%%%%%%%%%%%%%%%%%%%%%%%%%%%
% Problem 4
%%%%%%%%%%%%%%%%%%%%%%%%%%%%%%%%%%%%%%%%%%%%%%%%%%%%%%%%%%
\section*{Problem 4:  Linear Maps and Differentiation of polynomials}

\textbf{Solution}

\begin{enumerate}[(a)]
    \item 
    For any $p_1(x),\ p_2(x) \in \mathcal{P}_n$,
    \begin{equation}
        T(p_1(x)+p_2(x))=\frac{d(p_1(x)+p_2(x))}{dx}=\frac{dp_1(x)}{dx}+ \frac{dp_2(x)}{dx} = T(p_1(x))+T(p_2(x))
    \end{equation}
    \begin{equation}
        T(\alpha p_1(x))=\frac{d(\alpha p_1(x))}{dx}=\alpha \frac{dp_1(x)}{dx}= \alpha T(p_1(x))
    \end{equation}
    Since equations (3) and (4) are valid, $T$ is linear.
    \item 
    For any $p_(x)\in \mathcal{P}_n$, by using $\{1\ x\ x^2\ \ldots\ x^n\}$ as basis, $p_(x)$ could be represented as
    \begin{equation}
        p_(x)=\sum_{i=0}^{n}\alpha_i x^i
        = 
        \begin{bmatrix}
            1 & x & x^2 & \ldots & x^n
        \end{bmatrix}
        \begin{bmatrix}
            \alpha_0  \\
            \alpha_1  \\
            \vdots \\
            \alpha_n \\
        \end{bmatrix}\nonumber
    \end{equation}
    Similarily
    \begin{equation}
        T(p_(x))=
        \begin{bmatrix}
            1 & x & x^2 & \ldots & x^n
        \end{bmatrix}
        \begin{bmatrix}
            \alpha_1  \\
            2\alpha_2  \\
            \vdots \\
            n\alpha_n \\
            0
        \end{bmatrix}\nonumber
    \end{equation}
    So we can find a matrix $A\in \mathbb{R}^{(n+1)\times (n+1)}$ that transforms the $p_(x)$ coefficient matrix to $T(p_(x))$ coefficient matrix.
    It means that
    \begin{equation}
        \begin{bmatrix}
            \alpha_1  \\
            2\alpha_2  \\
            \vdots \\
            n\alpha_n \\
            0
        \end{bmatrix}=A
        \begin{bmatrix}
            \alpha_0  \\
            \alpha_1  \\
            \vdots \\
            \alpha_n \\
        \end{bmatrix}
        \nonumber
    \end{equation}
    Then
    \begin{equation}
        A=
        \begin{bmatrix}
            0 & 1 & 0 & \ldots & 0 & 0 \\
            0 & 0 & 2 & \ldots & 0 & 0  \\
            \vdots & \vdots & \vdots &   & \vdots & \vdots  \\
            0 & 0 & 0 & \ldots & n & 0  \\
            0 & 0 & 0 & \ldots & 0 & 0  \\
        \end{bmatrix}
        \nonumber
    \end{equation}
    Obviously $rank(A)=n$.
\end{enumerate}

%%%%%%%%%%%%%%%%%%%%%%%%%%%%%%%%%%%%%%%%%%%%%%%%%%%%%%%%%%
% Problem 5
%%%%%%%%%%%%%%%%%%%%%%%%%%%%%%%%%%%%%%%%%%%%%%%%%%%%%%%%%%
\section*{Problem 5:  Rank of $AA^T$}

\textbf{Solution}

\begin{enumerate}[(a)]
    \item 
    For any $A\in \mathbb{R}^{m\times n}$,
    \begin{equation}
        dim[R(AA^T)]+dim[N(AA^T)]=m
        \nonumber
    \end{equation}
    \begin{equation}
        dim[R(A^T)]+dim[N(A^T)]=m
        \nonumber
    \end{equation}
    Since $rank(A)=rank(A^T)=dim[R(A^T)]$, so if we can show that $dim[N(AA^T)]=dim[N(A^T)]$, then $rank(A)=rank(AA^T)$ is valid.
    The following will proof $dim[N(AA^T)]=dim[N(A^T)]$.\\
    For any $x\in N(A^T)$,
    \begin{equation}
        A^Tx=0 \Rightarrow A(A^Tx)=0 \Rightarrow AA^Tx=0
        \nonumber
    \end{equation}
    So for any $x\in N(A^T)$, $x$ also satisfies $x\in N(AA^T)$.
    Conversely, for any $x\in N(AA^T)$,
    \begin{equation}
        AA^Tx=0 \Rightarrow x(AA^Tx)=0 \Rightarrow xAA^Tx=0 \Rightarrow (A^Tx)^TA^Tx=0
        \nonumber
    \end{equation}
    According to the conclusion of problem3 (c), $A^Tx=0$. It means that for for any $x\in N(AA^T)$,
    $x\in N(A^T)$. Hence $N(A^T)=N(AA^T)$ and $dim[N(AA^T)]=dim[N(A^T)]$. Further we can get
    \begin{equation}
        dim[R(AA^T)]=m-dim[N(AA^T)]=m-dim[N(A^T)]=dim[R(A^T)]
        \nonumber
    \end{equation}
    Since $rank(A)=rank(A^T)=dim[R(A^T)]$ and $rank(AA^T)=dim[R(AA^T)]$, So
    \begin{equation}
        rank(A)rank(AA^T)
        \nonumber
    \end{equation}
    \item 
    The statement is invalid. Suppose
    \begin{equation}
        A=
        \begin{bmatrix}
            1 & -i \\
            1 & -i 
        \end{bmatrix},\ 
        A^T=
        \begin{bmatrix}
            1 & 1 \\
            -i & -i 
        \end{bmatrix}
        \nonumber
    \end{equation}
    Then
    \begin{equation}
        AA^T=
        \begin{bmatrix}
            0 & 0 \\
            0 & 0 
        \end{bmatrix}
        \nonumber
    \end{equation}
    Obviously $rank(AA^T)\neq rank(A)$.
    \item 
    Similarily to problem(a), we can show that $N(A^H)=N(AA^H)$. For any $A\in \mathbb{C}^{m\times n}$,
    if $x\in N(A^H)$, then
    \begin{equation}
        A^Hx=0 \Rightarrow AA^Hx=0 \Rightarrow x\in N(AA^H)
        \nonumber
    \end{equation}
    Conversely, if $x\in N(AA^H)$, then
    \begin{equation}
        AA^Hx=0 \Rightarrow xAA^Hx=0 \Rightarrow (A^Hx)^HA^Hx=0
        \nonumber
    \end{equation}
    Suppose $[A^Hx]_{jk}=a_{jk}+b_{jk}i$, it is easy to get that $[(A^Hx)^HA^Hx]_{jk}=a_{jk}^2+b_{jk}^2$. Hence
    if $(A^Hx)^HA^Hx=0$, then $A^Hx=0$. So if $x\in N(AA^H)$, then $x\in N(A^H)$. In conclusion, $N(A^H)=N(AA^H)$
    and $dim[N(A^H)]=dim[N(AA^H)]$. Since
    \begin{equation}
        dim[R(AA^H)]+dim[N(AA^H)]=m
        \nonumber
    \end{equation}
    \begin{equation}
        dim[R(A^H)]+dim[N(A^H)]=m
        \nonumber
    \end{equation}
    So
    \begin{equation}
        dim[R(A^H)]=dim[R(AA^H)] \Rightarrow rank[A]=rank[A^H]=rank[AA^H]
        \nonumber
    \end{equation}
\end{enumerate}

%%%%%%%%%%%%%%%%%%%%%%%%%%%%%%%%%%%%%%%%%%%%%%%%%%%%%%%%%%
% Problem 6
%%%%%%%%%%%%%%%%%%%%%%%%%%%%%%%%%%%%%%%%%%%%%%%%%%%%%%%%%%
\section*{Problem 6: Left and Right Inverses}

\textbf{Solution}

\begin{enumerate}[(a)]
    \item 
    
\end{enumerate}

\end{document}
