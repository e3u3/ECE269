%%%%%%%%%%%%%%%%% DO NOT CHANGE HERE %%%%%%%%%%%%%%%%%%%% {
\documentclass[12pt,letterpaper]{article}
\usepackage{fullpage}
\usepackage[top=2cm, bottom=4.5cm, left=2.5cm, right=2.5cm]{geometry}
\usepackage{amsmath,amsthm,amsfonts,amssymb,amscd}
\usepackage{lastpage}
\usepackage{enumerate}
\usepackage{fancyhdr}
\usepackage{mathrsfs}
\usepackage{xcolor}
\usepackage{graphicx}
\usepackage{listings}
\usepackage{hyperref}
\usepackage{cases}


\hypersetup{%
colorlinks=true,
linkcolor=blue,
linkbordercolor={0 0 1}
}

\setlength{\parindent}{0.0in}
\setlength{\parskip}{0.05in}
%%%%%%%%%%%%%%%%%%%%%%%%%%%%%%%%%%%%%%%%%%%%%%%%%%%%%%%%%% }

%%%%%%%%%%%%%%%%%%%%%%%% CHANGE HERE %%%%%%%%%%%%%%%%%%%% {
\newcommand\course{ECE 269}
\newcommand\semester{Fall 2019}
\newcommand\hwnumber{\#1}                 % <-- ASSIGNMENT #
\newcommand\NetIDa{Jiaming Lai}           % <-- YOUR NAME
\newcommand\NetIDb{A53314574}           % <-- STUDENT ID #
%%%%%%%%%%%%%%%%%%%%%%%%%%%%%%%%%%%%%%%%%%%%%%%%%%%%%%%%%% }

%%%%%%%%%%%%%%%%% DO NOT CHANGE HERE %%%%%%%%%%%%%%%%%%%% {
\pagestyle{fancyplain}
\headheight 35pt
\lhead{\NetIDa}
\lhead{\NetIDa\\\NetIDb}                 
\chead{\textbf{\Large Assignment \hwnumber}}
\rhead{\course \\ \semester}
\lfoot{}
\cfoot{}
\rfoot{\small\thepage}
\headsep 1.5em
%%%%%%%%%%%%%%%%%%%%%%%%%%%%%%%%%%%%%%%%%%%%%%%%%%%%%%%%%% }

\begin{document}

%%%%%%%%%%%%%%%%%%%%%%%%%%%%%%%%%%%%%%%%%%%%%%%%%%%%%%%%%%
% Problem 1
%%%%%%%%%%%%%%%%%%%%%%%%%%%%%%%%%%%%%%%%%%%%%%%%%%%%%%%%%%
\section*{Problem 1: Convolution as Linea Map}

\textbf{Solution}

\begin{enumerate}[(a)]
    \item 
    Obviously the size of matrice $T$ must be $(N+1)\times(N+1)$. Let $T_{i1},T_{i2}, \ldots ,T_{i(N+1)}$ be the i-th row of matrice $T$.
    Since $y=Tx$, we can get
    \begin{equation}
        v(i-1)=\sum_{j=1}^{N+1} T_{ij}u(j-1)=T_{i1}u(0)+T_{i2}u(1)+ \ldots +T_{i(N+1)}u(N) \label{ProblemF(1)}
    \end{equation}
    Accoring to the convolution function,
    \begin{equation}
        v(i-1)=\sum_{k=-\infty}^{+\infty}h(k)u(i-1-k)=\sum_{k=(i-1-N)}^{i-1}h(k)u(i-1-k) \label{ProblemF(2)}
    \end{equation}
    Comparing function (1) and function (2), it is easy to conclude that
    \begin{equation}
        T_{ij} = h(i-j), \ where\ 1\leq i \leq (N+1),\ 1\leq j \leq (N+1) \nonumber
    \end{equation}
    So the matrice is a matrice where each element could be represented as $T_{ij} = h(i-j), \ where\ 1\leq i \leq (N+1),\ 1\leq j \leq (N+1)$.
    \item 
    The structure of matrice $T$ could be described as follow:
    \begin{equation}
        T_{i,j} = T_{i+1,j+1} \nonumber
    \end{equation}
\end{enumerate}

%%%%%%%%%%%%%%%%%%%%%%%%%%%%%%%%%%%%%%%%%%%%%%%%%%%%%%%%%%
% Problem 2
%%%%%%%%%%%%%%%%%%%%%%%%%%%%%%%%%%%%%%%%%%%%%%%%%%%%%%%%%%
\section*{Problem 2: Adjacency graph}

\textbf{Solution}


%%%%%%%%%%%%%%%%%%%%%%%%%%%%%%%%%%%%%%%%%%%%%%%%%%%%%%%%%%
% Problem 3
%%%%%%%%%%%%%%%%%%%%%%%%%%%%%%%%%%%%%%%%%%%%%%%%%%%%%%%%%%
\section*{Problem 3: Vector Spaces of Polynomials}

\textbf{Solution}



%%%%%%%%%%%%%%%%%%%%%%%%%%%%%%%%%%%%%%%%%%%%%%%%%%%%%%%%%%
% Problem 4
%%%%%%%%%%%%%%%%%%%%%%%%%%%%%%%%%%%%%%%%%%%%%%%%%%%%%%%%%%
\section*{Problem 4: Symmetric and Hermitian matrices}
% If the Problem is divided into items, use "enumerate"

\textbf{Solution}



%%%%%%%%%%%%%%%%%%%%%%%%%%%%%%%%%%%%%%%%%%%%%%%%%%%%%%%%%%
% Problem 5
%%%%%%%%%%%%%%%%%%%%%%%%%%%%%%%%%%%%%%%%%%%%%%%%%%%%%%%%%%
\section*{Problem 5: Properties of Vector Spaces}

\textbf{Solution}


%%%%%%%%%%%%%%%%%%%%%%%%%%%%%%%%%%%%%%%%%%%%%%%%%%%%%%%%%%
% Problem 6
%%%%%%%%%%%%%%%%%%%%%%%%%%%%%%%%%%%%%%%%%%%%%%%%%%%%%%%%%%
\section*{Problem 6: Linear Independence}

\textbf{Solution}



%%%%%%%%%%%%%%%%%%%%%%%%%%%%%%%%%%%%%%%%%%%%%%%%%%%%%%%%%%
% Problem 7
%%%%%%%%%%%%%%%%%%%%%%%%%%%%%%%%%%%%%%%%%%%%%%%%%%%%%%%%%%
\section*{Problem 7: Finding Basis}

\textbf{Solution}



\end{document}
