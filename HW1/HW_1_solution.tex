%%%%%%%%%%%%%%%%% DO NOT CHANGE HERE %%%%%%%%%%%%%%%%%%%% {
\documentclass[12pt,letterpaper]{article}
\usepackage{fullpage}
\usepackage[top=2cm, bottom=4.5cm, left=2.5cm, right=2.5cm]{geometry}
\usepackage{amsmath,amsthm,amsfonts,amssymb,amscd}
\usepackage{lastpage}
\usepackage{enumerate}
\usepackage{fancyhdr}
\usepackage{mathrsfs}
\usepackage{xcolor}
\usepackage{graphicx}
\usepackage{listings}
\usepackage{hyperref}


\hypersetup{%
colorlinks=true,
linkcolor=blue,
linkbordercolor={0 0 1}
}

\setlength{\parindent}{0.0in}
\setlength{\parskip}{0.05in}
%%%%%%%%%%%%%%%%%%%%%%%%%%%%%%%%%%%%%%%%%%%%%%%%%%%%%%%%%% }

%%%%%%%%%%%%%%%%%%%%%%%% CHANGE HERE %%%%%%%%%%%%%%%%%%%% {
\newcommand\course{ECE 269}
\newcommand\semester{Fall 2019}
\newcommand\hwnumber{\#1}                 % <-- ASSIGNMENT #
\newcommand\NetIDa{Jiaming Lai}           % <-- YOUR NAME
\newcommand\NetIDb{A53314574}           % <-- STUDENT ID #
%%%%%%%%%%%%%%%%%%%%%%%%%%%%%%%%%%%%%%%%%%%%%%%%%%%%%%%%%% }

%%%%%%%%%%%%%%%%% DO NOT CHANGE HERE %%%%%%%%%%%%%%%%%%%% {
\pagestyle{fancyplain}
\headheight 35pt
\lhead{\NetIDa}
\lhead{\NetIDa\\\NetIDb}                 
\chead{\textbf{\Large Assignment \hwnumber}}
\rhead{\course \\ \semester}
\lfoot{}
\cfoot{}
\rfoot{\small\thepage}
\headsep 1.5em
%%%%%%%%%%%%%%%%%%%%%%%%%%%%%%%%%%%%%%%%%%%%%%%%%%%%%%%%%% }

\begin{document}

%%%%%%%%%%%%%%%%%%%%%%%%%%%%%%%%%%%%%%%%%%%%%%%%%%%%%%%%%%
% Problem 1
%%%%%%%%%%%%%%%%%%%%%%%%%%%%%%%%%%%%%%%%%%%%%%%%%%%%%%%%%%
\section*{Problem 1: Vector Spaces other than $\mathbb{R}^N$}

\begin{enumerate}[a)]
\item 
Suppose A is a set of rational numbers defined over $\mathbb{R}$. $A_1\in{A}$.
According to the definition of rational numbers, $A_1 = \frac{b}{a}$, where a and b
are integers and $a\neq{0}$. Suppose $B_1\in{\mathbb{R}}$, and 
\begin{equation}
    B_1 = \pi \cdot A_1 = \frac{b\pi}{a} \nonumber
\end{equation}
Obviously, $B_1$ is not a rational number, that is $B_1\notin{A}$.
So the set of rational numbers defined over $\mathbb{R}$ doesn't satisfy closure of scalar multification.
So the set is not valid vector field.

\item 
Suppose set $A=\{a_0+a_1x+a_2x^2|a_0,a_1,a_2\in\mathbb{R}^+\}$. $A_1\in{A}$ and $A_1=a_0+a_1x+a_2x^2$, $a_0,a_1,a_2\in\mathbb{R}^+$.
Let
\begin{equation}
    B_1 = (-1) \cdot A_1 = (-a_0)+(-a_1)x+(-a_2)x^2 \nonumber
\end{equation}
Obviously, $(-a_0), (-a_1), (-a_2) \notin \mathbb{R}^+$, so $B_1\notin{A}$. Hence set A doesn't satisfy closure of scalar multification.
So set A is not valid vector field.

\item 
\begin{enumerate}[i.]
    \item
    Let
    \begin{equation}
        A = (\alpha_1+\alpha_2) \cdot
        \begin{bmatrix} a \\ b \end{bmatrix} = \begin{bmatrix} (\alpha_1+\alpha_2)a \\ b  \end{bmatrix},
        \alpha_1,\alpha_2 \in \mathbb{R} \nonumber
    \end{equation}
    \begin{equation}
        B = \alpha_1 \cdot \begin{bmatrix} a \\ b \end{bmatrix} + \alpha_2 \cdot \begin{bmatrix} a \\ b \end{bmatrix}
        = \begin{bmatrix} (\alpha_1+\alpha_2)a \\ 2b  \end{bmatrix},
        \alpha_1,\alpha_2 \in \mathbb{R} \nonumber
    \end{equation}
    Because $B\neq{A}$, so this set doesn't satisfy the following vector space property:
    \begin{equation}
        (\alpha+\beta) \cdot v = \alpha \cdot v + \beta \cdot v, \alpha,\beta \in \mathcal{F}\ and\ v \in \mathcal{V} \nonumber
    \end{equation}
    So this set is not a valid vector field.

    \item
    Let $r = 1 \in \mathbb{R}$
    \begin{equation}
        r \cdot A = r \cdot \begin{bmatrix} a \\ b \end{bmatrix} = \begin{bmatrix} a \\ 0 \end{bmatrix}
        \neq A \nonumber
    \end{equation}
    So this set doesn't satisfy the following vector space property:
    \begin{equation}
        1 \cdot v = v, where\ 1 \in \mathcal{F}\ and\ v \in \mathcal{V} \nonumber
    \end{equation}
    So this set is not a valid vector field.

    \item
    Suppose $B=\begin{bmatrix} a \\ b \end{bmatrix}$ and $\alpha_1,\alpha_2 \in \mathbb{R}$. Let
    \begin{equation}
        A_1 = (\alpha_1+\alpha_2) \cdot B = (\alpha_1+\alpha_2) \cdot \begin{bmatrix} a \\ b \end{bmatrix}
        = \begin{bmatrix} (\alpha_1+\alpha_2)a \\ (\alpha_1+\alpha_2)b \end{bmatrix} \nonumber
    \end{equation}
    \begin{equation}
        A_2 = \alpha_1 \cdot B + \alpha_2 \cdot B = 0 \nonumber
    \end{equation}
    Because $A_1\neq{A_2}$, so this set doesn't satisfy the following vector space property:
    \begin{equation}
        (\alpha_1+\alpha_2) \cdot v = \alpha_1 \cdot v + \alpha_2 \cdot v, 
        where\ \alpha_1,\alpha_2 \in \mathcal{F}\ and\ v \in \mathcal{V} \nonumber
    \end{equation}
    So this set is not a valid vector field.

    \item
    Suppose
    \begin{equation}
        A=\begin{bmatrix} \alpha_1 \\ \beta_1 \end{bmatrix},
        B=\begin{bmatrix} \alpha_2 \\ \beta_2 \end{bmatrix},
        C=\begin{bmatrix} \alpha_3 \\ \beta_3 \end{bmatrix}  \nonumber
    \end{equation}
    Let:
    \begin{equation}
        A+(B+C) =\begin{bmatrix} \alpha_1 \\ \beta_1 \end{bmatrix} + (\begin{bmatrix} \alpha_2 \\ \beta_2 \end{bmatrix}+
        \begin{bmatrix} \alpha_3 \\ \beta_3 \end{bmatrix}) = 
        \begin{bmatrix} \alpha_1 - \alpha_2 + \alpha_3 \\ \beta_1 - \beta_2 + \beta_3 \end{bmatrix}) \nonumber
    \end{equation}
    \begin{equation}
        (A+B)+C =(\begin{bmatrix} \alpha_1 \\ \beta_1 \end{bmatrix} + \begin{bmatrix} \alpha_2 \\ \beta_2 \end{bmatrix})+
        \begin{bmatrix} \alpha_3 \\ \beta_3 \end{bmatrix} = 
        \begin{bmatrix} \alpha_1 - \alpha_2 - \alpha_3 \\ \beta_1 - \beta_2 - \beta_3 \end{bmatrix}) \nonumber
    \end{equation}
    Because $A+(B+C)\neq{(A+B)+C}$, so this set doesn't satisfy the following vector space property:
    \begin{equation}
        A+(B+C) = (A+B)+C, 
        where\ A,B,C \in \mathcal{V} \nonumber
    \end{equation}
    So this set is not a valid vector field.
\end{enumerate}
\end{enumerate}

%%%%%%%%%%%%%%%%%%%%%%%%%%%%%%%%%%%%%%%%%%%%%%%%%%%%%%%%%%
% Problem 2
%%%%%%%%%%%%%%%%%%%%%%%%%%%%%%%%%%%%%%%%%%%%%%%%%%%%%%%%%%
\section*{Problem 2: Adjacency graph}


%%%%%%%%%%%%%%%%%%%%%%%%%%%%%%%%%%%%%%%%%%%%%%%%%%%%%%%%%%
% Problem 3
%%%%%%%%%%%%%%%%%%%%%%%%%%%%%%%%%%%%%%%%%%%%%%%%%%%%%%%%%%
\section*{Problem 3: Vector Spaces of Polynomials}
\begin{enumerate}[a)]
    \item A vector space should satisfy properties (A1)-(A5) and (M1)-(M5)
    \begin{enumerate}[({A}1).]
        \item 
    \end{enumerate}
\end{enumerate}

%%%%%%%%%%%%%%%%%%%%%%%%%%%%%%%%%%%%%%%%%%%%%%%%%%%%%%%%%%
% Problem 4
%%%%%%%%%%%%%%%%%%%%%%%%%%%%%%%%%%%%%%%%%%%%%%%%%%%%%%%%%%
\section*{Problem 4: Symmetric and Hermitian matrices}
% If the Problem is divided into items, use "enumerate"
\begin{enumerate}[a)]
    \item 
    ``There is a student in Gryffindor who has taken all elective classes.''
    
    Solution:
        \begin{equation}
            \exists x \forall y \forall z ( H(x, \text{Gryffindor}) \wedge P(x,y) )
        \end{equation}
    where 
    \begin{itemize}
        \item[] $H(x,z)$ is ``$x$ is of $z$ house''
        \item[] $P(x, y)$ is ``$x$ has taken $y$,''
        \item[] the domain for $x$ consists of all students in Hogwarts
        \item[] the domain for $y$ consists of all elective classes,
        \item[] and the domain for $z$ consists of all Hogwarts houses.
    \end{itemize}
    
    \item 
    Give a direct proof of the theorem ``If $n$ is an odd integer, then $n^2$ is odd.''
    
    Solution:
    \begin{enumerate}[1.]
        \item 
        \begin{equation}
            \forall n(P(n) \rightarrow Q(n)),
        \end{equation}
        where
        \begin{itemize}
            \item[] $P(n)$ is ``$n$ is an odd integer'' and
            \item[] $Q(n)$ is ``$n^2$ is odd.''
        \end{itemize}
        
        \item 
        Assume $P(n)$  is true.
        
        \item 
        By definition, an odd integer is $n = 2k + 1$, 
        where $k$ is some integer.

        \item
        \begin{align*}
            n^2 &= (2k + 1)^2 \\
                &= 4k^2 + 4k + 1 \\
                &=  2(2k^2 + 2k) + 1
        \end{align*}
        
        \item 
        $\therefore n^2$ is an odd integer. $\qed$
    \end{enumerate}
    
    \item Let $A = \{1,2,3\}$ and $B = \{1,2,3,\{1,2,3\}\}$:
        
    Then, $A \in B$ and $A \subseteq B$.
    
    \item Let $A = \{1, 3, 5\}$, $B = \{1,2,3,\}$, and universe $U = \{1,2,3,4,5\}$:
    \begin{align*}
        A \cup B    &= \{1,2,3,5\}, \\
        A \cap B    &= \{1,3\}, \\
        A - B       &= \{5\},\\
        \bar{A}     &= \{2,4\},\\
        A - A       &= \emptyset .
    \end{align*}
\end{enumerate}

%%%%%%%%%%%%%%%%%%%%%%%%%%%%%%%%%%%%%%%%%%%%%%%%%%%%%%%%%%
% Problem 5
%%%%%%%%%%%%%%%%%%%%%%%%%%%%%%%%%%%%%%%%%%%%%%%%%%%%%%%%%%
\section*{Problem 5: Properties of Vector Spaces}

%%%%%%%%%%%%%%%%%%%%%%%%%%%%%%%%%%%%%%%%%%%%%%%%%%%%%%%%%%
% Problem 6
%%%%%%%%%%%%%%%%%%%%%%%%%%%%%%%%%%%%%%%%%%%%%%%%%%%%%%%%%%
\section*{Problem 6: Linear Independence}

%%%%%%%%%%%%%%%%%%%%%%%%%%%%%%%%%%%%%%%%%%%%%%%%%%%%%%%%%%
% Problem 7
%%%%%%%%%%%%%%%%%%%%%%%%%%%%%%%%%%%%%%%%%%%%%%%%%%%%%%%%%%
\section*{Problem 7: Finding Basis}
\end{document}
