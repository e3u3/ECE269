%%%%%%%%%%%%%%%%% DO NOT CHANGE HERE %%%%%%%%%%%%%%%%%%%% {
  \documentclass[12pt,letterpaper]{article}
  \usepackage{fullpage}
  \usepackage[top=2cm, bottom=4.5cm, left=2.5cm, right=2.5cm]{geometry}
  \usepackage{amsmath,amsthm,amsfonts,amssymb,amscd}
  \usepackage{lastpage}
  \usepackage{enumerate}
  \usepackage{fancyhdr}
  \usepackage{mathrsfs}
  \usepackage{xcolor}
  \usepackage{graphicx}
  \usepackage{listings}
  \usepackage{hyperref}
  
  \hypersetup{%
    colorlinks=true,
    linkcolor=blue,
    linkbordercolor={0 0 1}
  }
  
  \setlength{\parindent}{0.0in}
  \setlength{\parskip}{0.05in}
  %%%%%%%%%%%%%%%%%%%%%%%%%%%%%%%%%%%%%%%%%%%%%%%%%%%%%%%%%% }
  
  %%%%%%%%%%%%%%%%%%%%%%%% CHANGE HERE %%%%%%%%%%%%%%%%%%%% {
  \newcommand\course{ECE 269}
  \newcommand\semester{Fall 2019}
  \newcommand\hwnumber{\#1}                 % <-- ASSIGNMENT #
  \newcommand\NetIDa{Jiaming Lai}           % <-- YOUR NAME
  \newcommand\NetIDb{A53314574}           % <-- STUDENT ID #
  %%%%%%%%%%%%%%%%%%%%%%%%%%%%%%%%%%%%%%%%%%%%%%%%%%%%%%%%%% }
  
  %%%%%%%%%%%%%%%%% DO NOT CHANGE HERE %%%%%%%%%%%%%%%%%%%% {
  \pagestyle{fancyplain}
  \headheight 35pt
  \lhead{\NetIDa}
  \lhead{\NetIDa\\\NetIDb}                 
  \chead{\textbf{\Large Assignment \hwnumber}}
  \rhead{\course \\ \semester}
  \lfoot{}
  \cfoot{}
  \rfoot{\small\thepage}
  \headsep 1.5em
  %%%%%%%%%%%%%%%%%%%%%%%%%%%%%%%%%%%%%%%%%%%%%%%%%%%%%%%%%% }
  
  \begin{document}
  
  \section*{Problem 1}
  
  Answer to the problem goes here.
  Use a line per sentence.
  Leave a blank space to start a new paragraph. Next, an example typesetting mathematics in \LaTeX .
  
  %%%%%%%% Math formulas %%%%%%%%
  %%% For math formulas in text, put them between dollar signs '$'
  %% Example: 
  Showing that equation $a + b = \frac{c}{d}$ in evidence:
  %%% For stand alone math formulas use "align"
  %% Example: 
  \begin{equation}
      \label{eq_example}
      a + b = \frac{c}{d}
  \end{equation}
  
  Note that equation \ref{eq_example} was automatically numbered.
  If you prefer not numbered equations, see the next example.
  \begin{enumerate}[a)]
    \item 
    ``There is a student in Gryffindor who has taken all elective classes.''
  \end{enumerate}
  
  \section*{Example Problem 2}
  
  Showing that $\neg (p \rightarrow q)$ and  $p \wedge \neg q$ are logically equivalent.
  \begin{align*}
     \neg (p \rightarrow q)   & \equiv \neg ( \neg p \vee q ) \\ % \\ makes a new line
                              & \equiv \neg ( \neg p \vee q ) \\
                              & \equiv \neg ( \neg p ) \wedge \neg ( q ) \\
                              & \equiv p \wedge \neg q
  \end{align*}
  
  Note that $\&$ is where the equations align.
  
  \section*{Example Problem 3}
  
  Constructing the \emph{Truth Table} of $(p \rightarrow q) \wedge (\neg p \leftrightarrow q)$ in Table \ref{tb_truth_table}:
  
  \begin{table}[h]    % [h] means to print the table here
  \caption{Caption here. Leave it blank if you will not refer it.}
  \label{tb_truth_table}
      \centering  % to center the table https://www.overleaf.com/project/5d757e7e591aa30001b65c17
      \begin{tabular}{cc|c|cc|c} % one 'c' for each column. It means centered. You can use 'l' or 'r' for left and right, respectively. '|' prints a line
  
          $p$ &   $q$ &   $\neg p$    &   $p \rightarrow q$  &   $\neg p \leftrightarrow q$  &   $(p \rightarrow q) \wedge (\neg p \leftrightarrow q)$ \\ \hline
          T   &   T   &   F           &   T                   &   F                           &   F   \\
          T   &   F   &   F           &   F                   &   T                           &   F   \\
          F   &   T   &   T           &   T                   &   T                           &   T   \\
          F   &   F   &   T           &   T                   &   F                           &   F   
      \end{tabular}
  \end{table}
  
  
  \section*{Example Problem 4}
  % If the Problem is divided into items, use "enumerate"
  \begin{enumerate}[a)]
      \item 
      ``There is a student in Gryffindor who has taken all elective classes.''
      
      Solution:
          \begin{equation}
              \exists x \forall y \forall z ( H(x, \text{Gryffindor}) \wedge P(x,y) )
          \end{equation}
      where 
      \begin{itemize}
          \item[] $H(x,z)$ is ``$x$ is of $z$ house''
          \item[] $P(x, y)$ is ``$x$ has taken $y$,''
          \item[] the domain for $x$ consists of all students in Hogwarts
          \item[] the domain for $y$ consists of all elective classes,
          \item[] and the domain for $z$ consists of all Hogwarts houses.
      \end{itemize}
      
      \item 
      Give a direct proof of the theorem ``If $n$ is an odd integer, then $n^2$ is odd.''
      
      Solution:
      \begin{enumerate}[1.]
          \item 
          \begin{equation}
              \forall n(P(n) \rightarrow Q(n)),
          \end{equation}
         where
         \begin{itemize}
              \item[] $P(n)$ is ``$n$ is an odd integer'' and
              \item[] $Q(n)$ is ``$n^2$ is odd.''
         \end{itemize}
          
          \item 
          Assume $P(n)$  is true.
          
          \item 
          By definition, an odd integer is $n = 2k + 1$, 
          where $k$ is some integer.
  
          \item
          \begin{align*}
              n^2 &= (2k + 1)^2 \\
                  &= 4k^2 + 4k + 1 \\
                  &=  2(2k^2 + 2k) + 1
          \end{align*}
          
          \item 
          $\therefore n^2$ is an odd integer. $\qed$
      \end{enumerate}
      
      
  
      \item Let $A = \{1,2,3\}$ and $B = \{1,2,3,\{1,2,3\}\}$:
          
      Then, $A \in B$ and $A \subseteq B$.
      
      \item Let $A = \{1, 3, 5\}$, $B = \{1,2,3,\}$, and universe $U = \{1,2,3,4,5\}$:
      \begin{align*}
          A \cup B    &= \{1,2,3,5\}, \\
          A \cap B    &= \{1,3\}, \\
          A - B       &= \{5\},\\
          \bar{A}     &= \{2,4\},\\
          A - A       &= \emptyset .
      \end{align*}
  
  \end{enumerate}
  
  \end{document}
  