%%%%%%%%%%%%%%%%% DO NOT CHANGE HERE %%%%%%%%%%%%%%%%%%%% {
\documentclass[12pt,letterpaper]{article}
\usepackage{fullpage}
\usepackage[top=2cm, bottom=4.5cm, left=2.5cm, right=2.5cm]{geometry}
\usepackage{amsmath,amsthm,amsfonts,amssymb,amscd}
\usepackage{lastpage}
\usepackage{enumerate}
\usepackage{fancyhdr}
\usepackage{mathrsfs}
\usepackage{xcolor}
\usepackage{graphicx}
\usepackage{listings}
\usepackage{hyperref}
\usepackage{cases}


\hypersetup{%
colorlinks=true,
linkcolor=blue,
linkbordercolor={0 0 1}
}

\setlength{\parindent}{0.0in}
\setlength{\parskip}{0.05in}
%%%%%%%%%%%%%%%%%%%%%%%%%%%%%%%%%%%%%%%%%%%%%%%%%%%%%%%%%% }

%%%%%%%%%%%%%%%%%%%%%%%% CHANGE HERE %%%%%%%%%%%%%%%%%%%% {
\newcommand\course{ECE 269}
\newcommand\semester{Fall 2019}
\newcommand\hwnumber{\#4}                 % <-- ASSIGNMENT #
\newcommand\NetIDa{Jiaming Lai}           % <-- YOUR NAME
\newcommand\NetIDb{A53314574}           % <-- STUDENT ID #
%%%%%%%%%%%%%%%%%%%%%%%%%%%%%%%%%%%%%%%%%%%%%%%%%%%%%%%%%% }

%%%%%%%%%%%%%%%%% DO NOT CHANGE HERE %%%%%%%%%%%%%%%%%%%% {
\pagestyle{fancyplain}
\headheight 35pt
\lhead{\NetIDa}
\lhead{\NetIDa\\\NetIDb}                 
\chead{\textbf{\Large Assignment \hwnumber}}
\rhead{\course \\ \semester}
\lfoot{}
\cfoot{}
\rfoot{\small\thepage}
\headsep 1.5em
%%%%%%%%%%%%%%%%%%%%%%%%%%%%%%%%%%%%%%%%%%%%%%%%%%%%%%%%%% }

\begin{document}

%%%%%%%%%%%%%%%%%%%%%%%%%%%%%%%%%%%%%%%%%%%%%%%%%%%%%%%%%%
% Problem 1
%%%%%%%%%%%%%%%%%%%%%%%%%%%%%%%%%%%%%%%%%%%%%%%%%%%%%%%%%%
\section*{Problem 1: Moore–Penrose Pseudoinverse}

\textbf{Solution}

\begin{enumerate}[(a)]
    %%%%%%%%%%%
    % Item (a)
    %%%%%%%%%%%
    \item 
    Suppose $A^+$ is not unique and there is $A_1^+$ and $A_2^+$. According to the difinition,
    for any matrix $A\in \mathbb{R}^{m\times n}$,
    \begin{equation}
        A=AA_1^+A=A(A_1^+A)^T=AA^T(A_1^+)^T \nonumber
    \end{equation}
    \begin{equation}
        \Rightarrow AA^T(A_1^+-A_2^+)^T=0 \nonumber
    \end{equation}
    Recall the result of Homework\#2 problem 5(a), $N(AA^T)=N(A^T)$. Then we get
    \begin{equation}
        A^T(A_1^+-A_2^+)^T=0 \Rightarrow (A_1^+-A_2^+)A=0 \Rightarrow A_1^+A=A_2^+A \nonumber
    \end{equation}
    We can also prove $AA_1^+=AA_2^+$ in the similar way.
    Because $AA_1^+=AA_2^+$ and $A_1^+A=A_2^+A$,
    \begin{equation}
        A^+_1=A_1^+AA_1^+=A_1^+AA_2^+=A_2^+AA^+_2=A^+_2 \nonumber
    \end{equation}
    Hence $A^+$ must be unique.
    %%%%%%%%%%%
    % Item (b)
    %%%%%%%%%%%
    \item 
    Denote $(A^TA)^{-1}A^T$ as $A^+$. Because matrix $A$ is tall, hence $n \leq m$.
    \begin{equation}
        A^+A=(A^TA)^{-1}A^TA=(A^TA)^{-1}(A^TA)=I \nonumber
    \end{equation}
    So $(A^TA)^{-1}A^T$ a left inverse of matrix $A$.
    \begin{equation}
        AA^+A=A(A^+A)=AI=A \nonumber
    \end{equation}
    \begin{equation}
        A^+AA^+=(A^+A)A^+=IA^+=A^+ \nonumber
    \end{equation}
    $A^+A=I$, so $A^+A$ is symmetric. Meanwhile,
    \begin{equation}
        (AA^+)^T=(A^+)^TA^T=[(A^TA)^{-1}A^T]^TA^T=A(A^TA)^{-1}A^T=AA^+ \nonumber
    \end{equation}
    So $AA^+$ is symmetric. In conclusion, $(A^TA)^{-1}A^T$ is the pseudoinverse and a left inverse
    of matrix $A$.
    %%%%%%%%%%%
    % Item (c)
    %%%%%%%%%%%
    \item 
    Denote $A^T(AA^T)^{-1}$ as $A^+$.
    \begin{equation}
        AA^+=AA^T(AA^T)^{-1}=(AA^T)(AA^T)^{-1}=I \nonumber
    \end{equation}
    So $A^T(AA^T)^{-1}$ is a right inverse of matrix $A$.
    \begin{equation}
        AA^+A=(AA^+)A=IA=A \nonumber
    \end{equation}
    \begin{equation}
        A^+AA^+=A^+(AA^+)=A^+I=A^+ \nonumber
    \end{equation}
    $AA^+=I$, so $AA^+$ is symmetric. Meanwhile,
    \begin{equation}
        (A^+A)^T=A^T(A^+)^T=A^T[A^T(AA^T)^{-1}]^T=A^T(AA^T)^{-1}A=A^+A \nonumber
    \end{equation}
    so $A^+A$ is symmetric. In conclusion, $A^T(AA^T)^{-1}$ is the pseudoinverse
    and a right inverse of matrix $A$.
    %%%%%%%%%%%
    % Item (d)
    %%%%%%%%%%%
    \item 
    \begin{equation}
        AA^{-1}A=IA=A\ and\ A^{-1}AA^{-1}=IA^{-1}=A^{-1} \nonumber
    \end{equation}
    Also $AA^{-1}=A^{-1}A=I$ is symmetric. So in conclusion, $A^{-1}$ is
    the pseudoinverse of a full-rank square matrix $A$.
    %%%%%%%%%%%
    % Item (e)
    %%%%%%%%%%%
    \item 
    For a projection matrix $A$, $A^2=A$ and $A^T=A$. Hence
    \begin{equation}
        AAA=AA=A \nonumber
    \end{equation}
    Because $A^T=A$, so $AA$ is symmetric. In conclusion, $A$ is
    the pseudoinverse of itself for a projection matrix $A$.
    %%%%%%%%%%%
    % Item (f)
    %%%%%%%%%%%
    \item 
    \begin{equation}
        A^T(A^+)^TA^T=[AA^+A]^T=A^T \nonumber
    \end{equation}
    \begin{equation}
        (A^+)^TA^T(A^+)^T=[A^+AA^+]^T=(A^+)^T \nonumber
    \end{equation}
    Meanwhile,
    \begin{equation}
        [A^T(A^+)^T]^T=A^+A\ \Rightarrow\ symmetric \nonumber
    \end{equation}
    \begin{equation}
        [(A^+)^TA^T]^T=AA^+\ \Rightarrow\ symmetric \nonumber
    \end{equation}
    So in conclusion, $(A^T)^+=(A^+)^T$.
    %%%%%%%%%%%
    % Item (g)
    %%%%%%%%%%%
    \item 
    \begin{enumerate}[i.]
        \item
        \begin{equation}
            AA^T[(A^+)^TA^+]AA^T=A(A^+A)^TA^+AA^T=A(A^+A)A^+AA^T=AA^+AA^T=AA^T \nonumber
        \end{equation}
        \begin{equation}
            (A^+)^TA^+(AA^T)(A^+)^TA^+=(A^+)^TA^+A(A^+A)^TA^+
            =(A^+)^TA^+AA^+AA^+=(A^+)^TA^+ \nonumber
        \end{equation}
        Meanwhile
        \begin{equation}
            AA^T[(A^+)^TA^+]=A(A^+A)^TA^+=A(A^+A)A^+
            =AA^+ \Rightarrow symmetric \nonumber
        \end{equation}
        \begin{equation}
            [(A^+)^TA^+]AA^T=(A^+)^T(A^+A)^TA^T=(A^+AA^+)^TA^T
            =AA^+ \Rightarrow symmetric \nonumber
        \end{equation}
        Hence $(AA^T)^+=(A^+)^TA^+$.
        \item
        \begin{equation}
            A^TA[A^+(A^+)^T]A^TA=A^TAA^+(AA^+)^TA=A^TAA^+AA^+A=A^TA \nonumber
        \end{equation}
        \begin{equation}
            A^+(A^+)^T(A^TA)A^+(A^+)^T=A^+AA^+AA^+(A^+)^T
            =A^+(A^+)^T \nonumber
        \end{equation}
        Meanwhile
        \begin{equation}
            A^+(A^+)^T(A^TA)=A^+AA^+A
            =A^+A\Rightarrow symmetric \nonumber
        \end{equation}
        \begin{equation}
            (A^TA)A^+(A^+)^T=(A^+AA^+A)^T=(A^+A)^T
            =AA^+ \Rightarrow symmetric \nonumber
        \end{equation}
        Hence $(A^TA)^+=A^+(A^+)^T$.
    \end{enumerate}
    %%%%%%%%%%%
    % Item (h)
    %%%%%%%%%%%
    \item 
    \begin{enumerate}[i.]
        \item
        For any $y\in R(A^+)$, there exit a vector $x$. S.T.
        \begin{equation}
            y=A^+x=A^+AA^+x=(A^+A)^TA^+x=A^T(A^+)^TA^+x \nonumber
        \end{equation}
        Hence $y\in R(A^T)$ and $R(A^+) \subset R(A^T)$.
        For any $y\in R(A^T)$, there exit a vector $x$. S.T.
        \begin{equation}
            y=A^Tx=A^T(A^T)^+A^Tx=A^T(A^+)^TA^Tx=(A^+A)^TA^Tx=A^+AA^Tx \nonumber
        \end{equation}
        Hence $y\in R(A^+)$ and $R(A^T) \subset R(A^+)$. In conclusion, $R(A^T)=R(A^+)$.
        \item
        For any $x\in N(A^T)$, $A^Tx=0$, hence
        \begin{equation}
            A^+x=A^+AA^+x=A^+(AA^+)^Tx=A^+(A^+)^TA^Tx=0 \nonumber
        \end{equation}
        So $N(A^T)\subset N(A^+)$.
        For any $x\in N(A^+)$, $A^+x=0$, hence
        \begin{equation}
            A^Tx=A^T(A^+)^TA^Tx=A^T(AA^+)^Tx=A^TAA^+x=0 \nonumber
        \end{equation}
        So $N(A^+)\subset N(A^T)$. In conclusion, $N(A^T)=N(A^+)$.
    \end{enumerate}
    %%%%%%%%%%%
    % Item (i)
    %%%%%%%%%%%
    \item 
    First, both $AA^+$ and $A^+A$ are symmetric. Second,
    \begin{equation}
        P^2=AA^+AA^+=AA^+ \nonumber
    \end{equation}
    \begin{equation}
        Q^2=A^+AA^+A=A^+A \nonumber
    \end{equation}
    Hence $P$ and $Q$ are projection matrix.
    %%%%%%%%%%%
    % Item (j)
    %%%%%%%%%%%
    \item 
    \begin{enumerate}
        \item
        Recall the result of problem 5 in Homework 3, $y=Px$ is the projection of $x$
        onto $R(P)$. For $\forall y \in R(P)$, there must exists $x\in \mathbb{R}^m$ S.T.
        \begin{equation}
            y=AA^+x\ \Rightarrow\ y=A(A^+x) \nonumber
        \end{equation}
        Hence for $\forall y \in R(P)$, there must exists $z=A^+x\in \mathbb{R}^n$, S.T.
        $y=Az$. Hence $R(P) \subset R(A)$.\\

        For $\forall y \in R(A)$, there must exists $x\in \mathbb{R}^n$ S.T.
        \begin{equation}
            y=Ax\ \Rightarrow\ y=AA^+Ax=PAx \nonumber
        \end{equation}
        Hence $R(A) \subset R(P)$.\\
        
        So $R(A) = R(P)$ and $y=Px$ is the projection of $x$ onto $R(A)$.
        \item 
        Similarly, $y=Qx$ is the projection of $x$ onto $R(Q)$. 
        For $\forall y \in R(Q)$, there must exists $x\in \mathbb{R}^n$ S.T.
        \begin{equation}
            y=A^+Ax\ \Rightarrow\ y=A^+(Ax) \nonumber
        \end{equation}
        Hence for $\forall y \in R(Q)$, there must exists $z=Ax\in \mathbb{R}^m$, S.T.
        $y=A^+z$. Hence $R(Q)\subset R(A^+)=R(A^T)$.\\
        
        For $\forall y \in R(A^+)$, there must exists $x\in \mathbb{R}^m$ S.T.
        \begin{equation}
            y=A^+x\ \Rightarrow\ y=(A^+A)A^+x \nonumber
        \end{equation}
        Hence $R(A^+)=R(A^T)\subset R(Q)$.\\

        So $R(A^T) = R(Q)$ and $y=Qx$ is the projection  of $x$ onto $R(A^T)$.
    \end{enumerate}
    %%%%%%%%%%%
    % Item (k)
    %%%%%%%%%%%
    \item 
    Recall the result in problem (j), the projection matrix onto
    $R(A)$ is $P=AA^+$,
    \begin{equation}
        Ax^*=AA^+b \nonumber
    \end{equation}
    Hence $Ax^*$ is the orthogonal projection of $b$ onto $R(A)$.
    Hence $x^*=A^+b$ is a least-squares solution.
    %%%%%%%%%%%
    % Item (l)
    %%%%%%%%%%%
    \item 
    It is clear that $x^*=A^+b=A^+Ax$ is the orthogonal projection of x
    onto $R(A^+A)$. Hence $(x-x^*)\perp x^*$. Then
    \begin{equation}
        ||x||_2^2=<x^*+x-x^*,x^*+x-x^*>=||x^*||^2_2+||x-x^*||^2_2 \ge ||x^*||^2_2 \nonumber
    \end{equation}
    Hence $x^*=A^+b$ is the least
    norm solution.
\end{enumerate}

%%%%%%%%%%%%%%%%%%%%%%%%%%%%%%%%%%%%%%%%%%%%%%%%%%%%%%%%%%
% Problem 2
%%%%%%%%%%%%%%%%%%%%%%%%%%%%%%%%%%%%%%%%%%%%%%%%%%%%%%%%%%
\section*{Problem 2: Eigenvalues}

\textbf{Solution}

\begin{enumerate}[(a)]
    \item
    The characteristic polynomial of $A$ is
    \begin{equation}
        p(\lambda)=det(\lambda I-A)=(\lambda-\lambda _1)(\lambda-\lambda _2)\ldots
        (\lambda-\lambda _n) \nonumber
    \end{equation}
    Hence
    \begin{equation}
        p(\lambda=0)=det(-A)=(-1)^ndet(A)=(-1)^n\lambda _1 \lambda _2\ldots \lambda _n \nonumber
    \end{equation}
    So $det(A)=\lambda _1 \lambda _2\ldots \lambda _n$.
    \item
    Because
    \begin{equation}
        \lambda I-A^T=(\lambda I-A)^T \nonumber
    \end{equation}
    and
    \begin{equation}
        det((\lambda I-A)^T)=det(\lambda I-A) \nonumber
    \end{equation}
    So $det(\lambda I-A)=det(\lambda I-A^T)$. $A^T$ and $A$ have the same 
    characteristic polynomial. Hence the eigenvalues of $A^T$ and
    $A$ are the same.
    \item
    Give the fact that $Av=\lambda _{i}v$, where $i=1,2,\ldots,n$,
    \begin{equation}
        A^kv=A^{k-1}\lambda _iv=A^{k-2}\lambda _i^2v=\ldots=\lambda _i^kv \nonumber
    \end{equation}
    Hence $\lambda _i^k,\ i=1,2,\ldots,n$ are eigenvalues of matrix $A^k$.
    \item
    If matrix $A$ is invertible, then suppose matrix $A$ has a zero eigenvalue $\lambda$. 
    There must be a vector $v\neq 0$, S.T.
    \begin{equation}
        Av=\lambda v =0 \nonumber
    \end{equation}
    Obviously $v\in N(A)$. Because $A$ is invertible, so $dim(N(A))=0\Rightarrow N(A)={0}$. Hence
    $v=0$. But this contradicts the fact that $v\neq 0$. So if $A$ is invertible, it does not have
    a zero eigenvalue.\\

    If matrix $A$ does not have a zero eigenvalue, then there is no a vector $v\neq 0$ S.T.
    \begin{equation}
        Av=\lambda v,\ \lambda =0\ \Rightarrow\ Av=0 \nonumber
    \end{equation}
    The only solution to $Av=0$ is $v=0$, which means $dim[N(A)]=0$. Hence matrix $A$ is full-rank
    and invertible.\\

    In conclusion, $A$ is invertible if and only if it does not have a zero eigenvalue.
    \item
    Accoding to the definition,
    \begin{equation}
        Av=\lambda _iv\ \Rightarrow\ A^{-1}Av=\lambda _iA^{-1}v\ \Rightarrow\ 
        v=\lambda _iA^{-1}v\ \Rightarrow\ \lambda _i^{-1}v=A^{-1}v \nonumber
    \end{equation}
    Hence $\lambda _i^{-1},\ i=1,2,\ldots,n$ are eigenvalues of $A^{-1}$.
    \item
    It is clear that both $T^{-1}AT$ and $A$
    are square matrix and the same size. The characteristic polynomial of $T^{-1}AT$ is
    \begin{equation}
        \begin{split}
            det(T^{-1}AT-\lambda I)&=det(T^{-1}AT-\lambda T^{-1}IT) \\
            &=det[T^{-1}(A-\lambda I)T] \\
            &=det(T^{-1})det(A-\lambda I)det(T)
        \end{split}
        \nonumber
    \end{equation}
    Because $det(T^{-1})det(T)=det(T^{-1}T)=1$. Hence
    \begin{equation}
        det(T^{-1}AT-\lambda I)=det(A-\lambda I) \nonumber
    \end{equation}
    So $A$ and $T^{-1}AT$ have the same eigenvalues.
\end{enumerate}

%%%%%%%%%%%%%%%%%%%%%%%%%%%%%%%%%%%%%%%%%%%%%%%%%%%%%%%%%%
% Problem 3
%%%%%%%%%%%%%%%%%%%%%%%%%%%%%%%%%%%%%%%%%%%%%%%%%%%%%%%%%%
\section*{Problem 3: Trace}

\textbf{Solution}

\begin{enumerate}[(a)]
    \item
    The characteristic polynomial of $A$ is
    \begin{equation}
        p(\lambda)=det(\lambda I-A)=(\lambda-\lambda _1)(\lambda-\lambda _2)\ldots
        (\lambda-\lambda _n) \nonumber
    \end{equation}
    Then the coefficients of $\lambda ^{n-1}$ is the nagetive sum of eigenvalues.\\

    Consider the computation process of $det(\lambda I-A)$, the only term that contains
    $\lambda ^{n-1}$ is
    \begin{equation}
        \sigma (1,2,3,\ldots,n)(\lambda I-A)_{11}(\lambda I-A)_{22}\ldots(\lambda I-A)_{nn} \nonumber
    \end{equation}
    If we expand the equation above, it is easy to find that coefficients of $\lambda ^{n-1}$
    is the nagetive sum of diagonal entries. Hence
    \begin{equation}
        tr(A)=\sum_{i=1}\lambda _i \nonumber
    \end{equation}
    \item
    Using the result from problem 2(c), $\lambda _i^k,\ i=1,2,\ldots,n$
    are eigenvalues of matrix $A^k$. Hence
    \begin{equation}
        tr(A^k)=\sum_{i=1}^{n}\lambda _i^k \nonumber
    \end{equation}
\end{enumerate}

%%%%%%%%%%%%%%%%%%%%%%%%%%%%%%%%%%%%%%%%%%%%%%%%%%%%%%%%%%
% Problem 4
%%%%%%%%%%%%%%%%%%%%%%%%%%%%%%%%%%%%%%%%%%%%%%%%%%%%%%%%%%
\section*{Problem 4: More on Eigenvalues}

\textbf{Solution}

Use the Schwarz Triangularization Theorem, for the square matrix $A$,
$A$ could be trangularized by an unitary matrix:
\begin{equation}
    A=UTU^H\ ,where\ UU^H=U^HU=I \nonumber
\end{equation}
\begin{equation}
    \Rightarrow ||A||_F=||UTU^H||_F \nonumber
\end{equation}
Hence $||A||_F=\sum_{i=1}^{n}\sum_{j=1}^{n}|A_{ij}|^2=||UTU^H||_F$.\\

As for the $||UTU^H||_F$, $||UTU^H||_F=tr[(UTU^H)^HUTU^H]=tr(UT^HTU^H)$,
\begin{equation}
    tr(UT^HTU^H)=\sum_{i}\sum_{j}\sum_{k}U_{ij}(T^HT)_{jk}U^H_{ki}=\sum_{j}\sum_{k}\sum_{i}(T^HT)_{jk}U^H_{ki}U_{ij}=tr(T^HTU^HU)=tr(T^HT) \nonumber
\end{equation}
\begin{equation}
    tr(T^HT)=\sum_{i}\sum_{j}T^H_{ij}T_{ji} \ge \sum_{i}T^H_{ii}T_{ii} \nonumber
\end{equation}
Because the diagonal elements of $T$ is the eigenvalues of $A$, hence $\sum_{i}T^H_{ii}T_{ii}=\sum_{i=1}^{n}|\lambda_i|^2$.
In conclusion, $\sum_{i=1}^{n}\sum_{j=1}^{n}|A_{ij}|^2=||A||_F=||UTU^H||_F \ge \sum_{i=1}^{n}|\lambda_i|^2$.

%%%%%%%%%%%%%%%%%%%%%%%%%%%%%%%%%%%%%%%%%%%%%%%%%%%%%%%%%%
% Problem 5
%%%%%%%%%%%%%%%%%%%%%%%%%%%%%%%%%%%%%%%%%%%%%%%%%%%%%%%%%%
\section*{Problem 5: Limit}

\textbf{Solution}
\begin{equation}
    det(\lambda I-A)=\lambda ^2 - 1.2\lambda +0.2=(\lambda -1)(\lambda -0.2) \nonumber
\end{equation}
Hence the eigenvalues of matrix $A$ is $\lambda _1=0.2$ and $\lambda _2=1$.
\begin{equation}
    A-\lambda _2I=
    \begin{bmatrix}
        4 & -1.6 \\
        12 & -4.8
    \end{bmatrix}
    \Rightarrow
    N(A-\lambda _2I)=span\left\{\begin{bmatrix}
        0.4 \\
        1
    \end{bmatrix}\right\} \nonumber
\end{equation}
\begin{equation}
    A-\lambda _1I=
    \begin{bmatrix}
        4.8 & -1.6 \\
        12 & -4
    \end{bmatrix}
    \Rightarrow
    N(A-\lambda _1I)=span\left\{\begin{bmatrix}
        1 \\
        3
    \end{bmatrix}\right\} \nonumber
\end{equation}
\begin{equation}
    \Rightarrow A=P
    \begin{bmatrix}
        \lambda _1 & 0 \\
        0 & \lambda _2
    \end{bmatrix}
    P^{-1} \nonumber
\end{equation}
where
\begin{equation}
    P=\begin{bmatrix}
        1 & 0.4 \\
        3 & 1
    \end{bmatrix} \nonumber
\end{equation}
Hence
\begin{equation}
    \lim_{n\to\infty} A^n=P
    \begin{bmatrix}
        \lambda _1^n & 0 \\
        0 & \lambda _2^n
    \end{bmatrix}
    P^{-1}
    =P
    \begin{bmatrix}
        0 & 0 \\
        0 & 1
    \end{bmatrix}
    P^{-1}
    =\begin{bmatrix}
        6 & -2 \\
        15 & -5
    \end{bmatrix} \nonumber
\end{equation}
\end{document}
